
% Copyright 2007, 2008, 2009 Elsevier Ltd 
% 
% This file is part of the 'Elsarticle Bundle'.
% --------------------------------------------- 
%
% It may be distributed under the conditions of the LaTeX Project Public
% License, either version 1.2 of this license or (at your option) any
% later version.  The latest version of this license is in
%    http://www.latex-project.org/lppl.txt
% and version 1.2 or later is part of all distributions of LaTeX
% version 1999/12/01 or later c.
%
% The list of all files belonging to the 'Elsarticle Bundle' is
% given in the file `manifest.txt'.
%
 
% Template article for Elsevier's document class `elsarticle'
% with harvard style bibliographic references
% SP 2008/03/01
%
%
%
% $Id: elsarticle-template-harv.tex 4 2009-10-24 08:22:58Z rishi $
%
%
%\documentclass[preprint,authoryear,12pt]{elsarticle}
\documentclass[preprint,12pt]{elsarticle}

% Use the option review to obtain double line spacing
%\documentclass[authoryear,preprint,review,12pt]{elsarticle}

% Use the options 1p,twocolumn; 3p; 3p,twocolumn; 5p; or 5p,twocolumn
% for a journal layout:
%\documentclass[final,authoryear,1p,times]{elsarticle}
%\documentclass[final,authoryear,1p,times,twocolumn]{elsarticle}
%\documentclass[final,authoryear,3p,times]{elsarticle}
%\documentclass[final,authoryear,3p,times,twocolumn]{elsarticle}
%\documentclass[final,authoryear,5p,times]{elsarticle}
%\documentclass[final,authoryear,5p,times,twocolumn]{elsarticle}

%% if you use PostScript figures in your article
%% use the gra

%%graphis package for simple commands
%% \usepackage{graphics}
%\usepackage{cases}
%% or use the graphicx package for more complicated commands
\usepackage{graphicx}
%% or use the epsfig package if you prefer to use the old commands
\usepackage{epsfig}
%\usepackage{subfig}
\usepackage{comment}

\usepackage{epstopdf}
\usepackage{pdflscape}
\usepackage{bm}
\usepackage{hyperref,url}
\hypersetup{colorlinks=true, urlcolor=blue, linkcolor=blue, citecolor=red}

%% The amssymb package provides various useful mathematical symbols

\usepackage{amssymb,amsmath,array}
%The amsthm package provides extended theorem environments

\usepackage{amsthm}
\usepackage{graphicx}
%\usepackage{subfigure}
  
%% The lineno packages adds line numbers. Start line numbering with
%% \begin{linenumbers}, end it with \end{linenumbers}. Or switch it on
%% for the whole article with \linenumbers after \end{frontmatter}.
%% \usepackage{lineno}

\usepackage{lscape}

%% natbib.sty is loaded by default. However, natbib options can be
%% provided with \biboptions{...} command. Following options are
%% valid:

%%   round  -  round parentheses are used (default)
%%   square -  square brackets are used   [option]
%%   curly  -  curly braces are used      {option}
%%   angle  -  angle brackets are used    <option>
%%   semicolon  -  multiple citations separated by semi-colon (default)
%%   colon  - same as semicolon, an earlier confusion
%%   comma  -  separated by comma
%%   authoryear - selects author-year citations (default)
%%   numbers-  selects numerical citations
%%   super  -  numerical citations as superscripts
%%   sort   -  sorts multiple citations according to order in ref. list
%%   sort&compress   -  like sort, but also compresses numerical citations
%%   compress - compresses without sorting
%%   longnamesfirst  -  makes first citation full author list
%%
%% \biboptions{longnamesfirst,comma}

% \biboptions{}

\newcommand{\JGnote}[1]{\fbox{\parbox{\textwidth}{ \color{red} JG Note $\Rightarrow$ #1}}}
\newcommand{\KCnote}[1]{\fbox{\parbox{\textwidth}{ \color{black} KC Note $\Rightarrow$ #1}}}
\newcommand{\red}{\textcolor{red}}
\newcommand{\blue}{\textcolor{blue}}
\newcommand{\green}{\textcolor{green}}
\newcommand{\frc}{\displaystyle\frac}
\newcommand{\PN}[2][error]{P$_{#1}$DG-P$_{#2}$}
\newcommand{\PNDG}[2][error]{P$_{#1}$DG-P$_{#2}$DG}
\newcommand{\eg}{{\it e.g., }} 
\newcommand{\ie}{{\it i.e., }}

\journal{Applied Mathematical Modelling}

\begin{document}

\begin{frontmatter}

%% Title, authors and addresses

%% use the tnoteref command within \title for footnotes;
%% use the tnotetext command for the associated footnote;
%% use the fnref command within \author or \address for footnotes;
%% use the fntext command for the associated footnote;
%% use the corref command within \author for corresponding author footnotes;
%% use the cortext command for the associated footnote;
%% use the ead command for the email address,
%% and the form \ead[url] for the home page:
%%
%% \title{Title\tnoteref{label1}}
%% \tnotetext[label1]{}
%% \author{Name\corref{cor1}\fnref{label2}}
%% \ead{email address}
%% \ead[url]{home page}
%% \fntext[label2]{}
%% \cortext[cor1]{}
%% \address{Address\fnref{label3}} 
%% \fntext[label3]{}

\title{A Reduced Order Model for Permeability Fields Using Singular Value Decomposition, SVD, by Interpolating in the Principal Component Spaces}
\author[UoA]{B. Lashore\corref{cor1}}\ead{lashorebabatunde@yahoo.com} \author[UoA]{J.L.M.A. Gomes} \author[UoA]{K. Christou}
\cortext[cor1]{Corresponding author.}
\address[UoA]{Mechanics of Fluids, Soils \& Structures Group, School of Engineering, University of Aberdeen, UK}


\begin{abstract}
  BlaBlaBla
\end{abstract}



\begin{keyword} %% keywords here, in the form: keyword \sep keyword
Singular Value Decomposition, SVD\sep Upscaling\sep Model Order Reduction, MOD \sep  Principal Component Spaces \sep Permeability field
\end{keyword}
 
\end{frontmatter}

%%%%%%%%%%%%%%%%%%%%%%%%%%%%%%%%%%%%%%%%%%%%%%%%%%%%%%%%%%%%%%%%%%%%%%%%%%%%%%%%%%%%%%%%%%%%%%%%%%%%%%%%%%%%%%%%%%%%%%%%%%%%%%%%%%%%%%%%%%%%%%%%%%%%%%%%%%%%%%%%%%%%%%%%%%%%%%%%%%%%%%%%%%%%%%%%%%% 
\section{Introduction}\label{section:intro}
Naturally occuring rock formations are inherently heterogeneous, which means that the rock properties such as permeability, pore spaces ({\it i.e.}, porosity) and pore throat vary from point to point, and at different length scales. The heterogeneity of rock formations (studied under the general title of porous media) is relevant in calculations required for understanding transportation and storage of fluids in porous media. This is because the heterogeneous properties are important parameters in the mathematical models used for the aforementioned calculations. Therefore, the representation of these heterogeneous properties is of great importance to the porous media and applied mathematics community as well as the oil and gas industry, the Carbon Capture and Storage (CCS) industry, and waste management industry for whom this has direct application in reservoir management, CO$_2$ transportation and storage in subsurface rocks and remediation of contaminated soil respectively.

Furthermore, due to limitations in computing resources, for simulations with very fine grids it is neither efficient nor possible to obtain exact values of geological properties at all points in the domain (\ie scale problems) \cite{Renard_1997} \cite{miller_1998} \cite{chen_2006}. This is particularly true for large geological domains which span several kilometres. Additionally, it is not possible to obtain reliable and accurate spatial information about geological properties throughout the domain ({\it i.e.}, uncertainty problems). It is possible to overcome these challenges by upscaling. Upscaling techniques solve both problems by replacing discrete geological and fluid property values of detailed high resolution domains with coarse descriptions (low resolution) of these properties \cite{Vereecken_2007}. And upscaling techniques which best preserve the statistical properties and flow dynamics behaviour of the high resolution domain gives the best representation of the heterogeneous property.

There are several upscaling techniques which fall under different categories \cite{Renard_1997, Szymkiewicz_2013, Hasting_2001}, however over the last decade, techniques classed as stochastic have received great attention from academic and industrial porous media communities worldwide \cite{Verwoerd_2009, Ravalec-Dupin_2010, Guilleminot_2012}. One of the reasons for this is because stochastic techniques are robust enough to handle multiphase flow in porous media, and they address scale and uncertainity problems which were previously mentioned.

This work couples stochastic and deterministic upscaling representations of the permeability field to a novel high-order accurate control volume finite element method (CVFEM) (see \cite{Gomes_2017} for further details of the model formulation) so as to investigate statistical properties and multiphase flow behaviour in highly heterogeneous porous media. 

Four upscaling techniques are investigated in this research. The arithmetic and harmonic mean are used to obtain the first two upscaled representation and these two are classed as deterministic techniques. The third is a randomly generated permeability field with a Guassian distribution, prescribed by a probability density function (PDF) which was obtained from a base case ({\it i.e.}, high-resolution mesh with known permeability distribution). The four upscaling technique is the crux of this research, it introduces singular value decomposition (SVD) as a method of upscaling using interpolation within the concept of principal component space to reduce the order of the permeability field.

In section 2, the basic mathematical explanation of SVD and its associated linear algebra properties are presented. The the concept of spaces is explained from the engineer's perspective and also from the mathematician's perspective. The objective of this explanation is to assist both parties in visualizing the relevant spaces and performing the necessary transformation within those spaces. Finally, the section describes model order space reduction for the permeability field within the principal component space. Section 3, starts by describing the high resolution base case on which the four upscaling techniques are modelled. The the simulation run which is similar for all the cases/models is described. A brief summary of the pre-processing step for each of the models is also presented. Section 4 provides the results and some discussion on the results of the simulations. It als gives further intepretation to the result obtained from the SVD upscaling technique. Finally, section 5 provides the conclusion.


\section{Singular Value Decomposition, SVD, and its application for reducing the order of a permeability field}\label{section:svd}
\subsection{A Brief Introduction to SVD}\label{subsection:svd_brief}

Singular Value Decomposition, SVD, exists within the linear algebra branch of mathematics as one of the most exciting means of factorizing a matrix, $\mathbf{A}$, into three parts $\mathbf{U}$, $\Sigma$ and $\mathbf{V}$. Where $\mathbf{U}$ and $\mathbf{V}$ are orthonormal, and $\Sigma$ is a diagonal matrix. If $\mathbf{A}$, is a square matrix then $\mathbf{U}$, $\Sigma$ and $\mathbf{V}$ are square matrix and $\mathbf{U}^{\intercal}\mathbf{U} = \mathbf{V}^{\intercal}\mathbf{V} = \mathbf{I}$. But if $\mathbf{A}$ is an $m * n$ matrix, $\mathbf{U}$ and $\mathbf{V}$ still remain as square matrices, albeit with different matricial sizes. $\mathbf{U}$ is an $m * m$ while $\mathbf{V}$ is an $n * n$. Another interesting property of $\mathbf{U}$ and $\mathbf{V}$ since they are square (irrespective of whether A is square or rectangular) is that $\mathbf{U}^{-1} = \mathbf{U}^{\intercal}$ and $\mathbf{V}^{-1} = \mathbf{V}^{\intercal}$. When $\mathbf{A}$ is an $m * n$ matrix, $\Sigma$ is a diagonal $m x n$ matrix. The diagonal entries of $\Sigma$ (in both the case of square and rectangular matrices) are referred to as the singular values of $\mathbf{A}$, and are usually denoted by sigma ($\sigma_{1}, \sigma_{2},..., \sigma_{r}$). For a rectangular matrix, the singular values fill the first $r$ places on the main diagonal of $\Sigma$, where $r$ is the rank of A. Additionally, the $i$th singular value, $\sigma_{i}$ is ordered such that $\sigma_{1} \geq \sigma_{2} \geq ...\geq \sigma_{r}$. A huge benefit of the SVD is the orthogonality of the columns in $\mathbf{U}$ and $\mathbf{V}$. The columns of $\mathbf{U}$ are known as the left singular vectors while the columns in $\mathbf{V}$ are known as the right singular vectors. A simple consideration of the benefit of this, is the ease with which it allows the inverse of square matrix $\mathbf{A}$ to be calculated. This example is presented thus:

\begin{equation}
 \mathbf{A}x = b
\end{equation}
\begin{equation}
 \mathbf{U} \Sigma \mathbf{V}^{\intercal} x = b
\end{equation}
\begin{equation}
 x = \mathbf{V} \Sigma^{-1} \mathbf{U}^{\intercal} b
\end{equation}

\section{Simulation Setup}\label{section:simulation_setup}


Four upscaled models were obtained from a base case which is a high-resolution permeability field with a known permeability distribution. The base case consists of four regions, where each region is uniformly randomly generated within a specified range. For the simulations, each of the model's domain is initially fully saturated with a fluid (Phase 2), and a wetting phase (1) fluid is driven into the domain from the left hand side at a constant mass flow rate. Newmann ({\it i.e.} no flux) boundary conditions were imposed on the upper and lower borders of each domain, while mixed fluids are recovered from the right-hand face. Fig (...........)a shows the permeability field of the base case while fig

\section{Results and Discussion}\label{section:results_discussion}

\section{Conclusion}\label{section:conclusion}




%\section{Acknowledgements}
%Mr William Rad\"unz would like to acknowledge the support from the Brazilian Research Council (CNPq) under the \textit{Science without Borders scholarship programme}. Mr Konstantinos Christou would like to acknowledge the support of the University of Aberdeen - College of Physical Science as well as the Aberdeen Formation Evaluation Society (\textit{AFES} is an SPWLA chapter). 

\clearpage 
%% References with bibTeX database:
\bibliographystyle{elsarticle-harv} 
%\bibliographystyle{elsarticle-num}
%\bibliographystyle{apacite}
\bibliography{references}
  
\clearpage 

\listoftables
\clearpage
%
\begin{landscape}
\begin{table}
  \begin{tabular}{c | c c  c  c  c  c  c  c  c  c   c}
    \hline
      {\bf Section} & $\phi$ & VR  & $S^{0}_{w}$ & $S^{0}_{nw}$ & $K_{1}$ & $K_{2}$ & $K_{3}$ & $K_{4}$ & $S_{w,irr}$ & $S_{nw,r}$ & $u^{0}_{w}$ \\ 
    \hline
     \ref{section:results_validation} & 0.2 & 1.0 & 0.0 & 1.0 & 2.5 & 1.0 & N/A & N/A & 0.2 & 0.3 & 1.0 \\
     \ref{section:results_homo_hete}(homogenous)  & 0.2 & 10.0 & 0.0 & 1.0 & 1.0 & N/A & N/A & N/A & 0.2 & 0.3 & XX  \\
     \ref{section:results_homo_hete}(heterogenous) & 0.2 & 1.0 & 0.0 & 1.0 & 1.e-11 -- & 1.e-12 -- & 1.e-12 -- & 1.e-10 & 0.2  & 0.3 & 1.0 \\
      &   &  &  &  & 5.e-10 & 5.e-10 & 1.e-10  &  &  &  & \\
     \ref{section:results_hete_fix_adapt}  & 0.2 & 1.0 & 0.0 & 1.0  & 3.0 & 2.0 & 5.0 & 1.0 & 0.2 & 0.3 & XX  \\
     \ref{section:results_dinlet}  & 0.2 & 1.0 & 0.0 & 1.0 & 3.0 & 2.0 & 5.0 & 1.0 & 0.2  & 0.3 & XX  \\
     \hline
   \end{tabular}
   \caption{Sumary of model set-up used in the numerical simulations. Superscript $0$ denotes initial condition. \blue{(PLEASE: 1. DOUBLE CHECK THE K, u and S0 VALUES FROM THE MPML FILES; 2. \underline{REPLACE ALL \red{XX}}; 3. COMPLETE THE TABLE FOR ALL SECTIONS/SIMULATIONS.  $S_{w,irr}$ and $S_{nw,r}$ are the same for all simulations.) }}\label{table:setup}
\end{table}
\end{landscape}
\clearpage










\clearpage  
\listoffigures
\clearpage


%%%%
%%%%  FIGURE
%%%%
%\begin{landscape}
\begin{figure}[ht] 
\vbox{\vspace{-1cm}
\centering
\includegraphics[width=.56\textwidth]{./Pics/Introduction/Graphical-illustration-of-model-order-reduction.png}
\vspace{0.cm}
\vspace{0.5cm}
}   
\caption{Graphical illustration of model order reduction (initially used by \citet{Schilders2008}, with graphics credited to Harvard University, Microsoft Research.)}
\vspace{1.5cm}
\label{fig:IllustrationMOR}
\end{figure}

\begin{figure}[ht] 
\vbox{\vspace{-1cm}
\hbox{\includegraphics[width=.56\textwidth]{./Pics/BaseCase/BaseCase_MeshOnly.png}
      \includegraphics[width=.56\textwidth]{./Pics/ArithMeanCase/ArithMeanCase_MeshOnly.png}}
\vspace{0.cm}
\hbox{\hspace{0.25cm} (a) High Resolution Grid (BaseCase) \hspace{0.75cm} (b) Low Resolution Grid (Upscaled Cases) \hspace{3.0cm}}
\vspace{0.5cm}
}   
\caption{Mesh grid used in the performed numerical simulations: (a) high-resolution and (b) upscaled cases with 4112 and 728 triangular elements respectively}
\label{fig:HiRes_LowRes_Mesh}
\end{figure}


%\end{landscape}
\clearpage


%%%%
%%%%  FIGURE
%%%%
\begin{landscape}
\begin{figure}[ht] 
\vbox{\vspace{-1cm}
\hspace{0.0cm} \hbox{\includegraphics[width=.56\textwidth]{./Pics/BaseCase/BaseCase_PermField_withMesh.png}
      \includegraphics[width=.56\textwidth]{./Pics/BaseCase/BaseCase_PermField_withoutMesh2.png}
      \includegraphics[width=.56\textwidth]{./Pics/ArithMeanCase/ArithMeanCase_PermField_withoutMesh2.png}}
\vspace{0.cm}
\hbox{\hspace{0.5cm} (a) BaseCase (overlapped with the mesh) \hspace{1.0cm} (b) BaseCase (without Mesh) \hspace{3.0cm} (c) ArithMean Case}
\vspace{0.5cm}
\hbox{
      \includegraphics[width=.56\textwidth]{./Pics/HarmMeanCase/HarmMeanCase_PermField_withoutMesh2.png}
      \includegraphics[width=.56\textwidth]{./Pics/PDFCase/PDFCase_PermField_withoutMesh2.png} 
      \includegraphics[width=.56\textwidth]{./Pics/SVDCase/SVDCase_PermField_withoutMesh2.png}}
\vspace{0.cm}
\hbox{ \hspace{1.5cm} (d) HarmMean Case \hspace{4.75cm} (e) PDFCase  \hspace{5.0cm} (f) SVDCase}
\vspace{0.cm}
}   
\caption{Permeability field for the base case as well as the upscaled cases (Permeability legend in Fig.~\ref{fig:PermFields}(a) is representative for all the other figures - \ie Fig.~\ref{fig:PermFields}(b) to (f))}
\label{fig:PermFields}
\end{figure}
\end{landscape}
\clearpage


%%%%
%%%%  FIGURE
%%%%
\begin{landscape}
\begin{figure}[ht] 
\vbox{\vspace{-1cm}
\hspace{0.0cm} \hbox{\includegraphics[width=.56\textwidth]{./Pics/BaseCase/BaseCase_Saturation_t_dot15.png}
      \includegraphics[width=.56\textwidth]{./Pics/BaseCase/BaseCase_Saturation_t_dot30.png}
      \includegraphics[width=.56\textwidth]{./Pics/BaseCase/BaseCase_Saturation_t_dot50.png}}
\vspace{0.cm}
\hbox{\hspace{0.5cm} (a) Phase 1 Saturation at t=0.15s \hspace{3.75cm} (b) t=0.30s \hspace{5.0cm} (c) t=0.50s}
\vspace{0.5cm}
\hbox{
      \includegraphics[width=.56\textwidth]{./Pics/BaseCase/BaseCase_Saturation_t_1dot15.png}
      \includegraphics[width=.56\textwidth]{./Pics/BaseCase/BaseCase_Saturation_t_1dot75.png} 
      \includegraphics[width=.56\textwidth]{./Pics/BaseCase/BaseCase_Saturation_t_2dot95.png}}
\vspace{0.cm}
\hbox{ \hspace{2.5cm} (d) t=1.15s \hspace{5.5cm} (e) t=1.75s   \hspace{5.5cm} (f) t=2.95s}
\vspace{0.cm}
}   
\caption{Simulations for the BaseCase (Saturation legend in Fig.~\ref{fig:BaseCase_Saturation}(a) is representative for all the other figures - \ie Fig.~\ref{fig:BaseCase_Saturation}(b) to (f))}
\label{fig:BaseCase_Saturation}
\end{figure}
\end{landscape}
\clearpage


%%%%
%%%%  FIGURE
%%%%
\begin{landscape}
\begin{figure}[ht] 
\vbox{\vspace{-1cm}
\hspace{0.0cm} \hbox{\includegraphics[width=.56\textwidth]{./Pics/ArithMeanCase/ArithMeanCase_Saturation_t_dot15.png}
      \includegraphics[width=.56\textwidth]{./Pics/ArithMeanCase/ArithMeanCase_Saturation_t_dot30.png}
      \includegraphics[width=.56\textwidth]{./Pics/ArithMeanCase/ArithMeanCase_Saturation_t_dot50.png}}
\vspace{0.cm}
\hbox{\hspace{0.5cm} (a) Phase 1 Saturation at t=0.15s \hspace{3.75cm} (b) t=0.30s \hspace{5.0cm} (c) t=0.50s}
\vspace{0.5cm}
\hbox{
      \includegraphics[width=.56\textwidth]{./Pics/ArithMeanCase/ArithMeanCase_Saturation_t_1dot15.png}
      \includegraphics[width=.56\textwidth]{./Pics/ArithMeanCase/ArithMeanCase_Saturation_t_1dot75.png} 
      \includegraphics[width=.56\textwidth]{./Pics/ArithMeanCase/ArithMeanCase_Saturation_t_2dot95.png}}
\vspace{0.cm}
\hbox{ \hspace{2.5cm} (d) t=1.15s \hspace{5.5cm} (e) t=1.75s   \hspace{5.5cm} (f) t=2.95s}
\vspace{0.cm}
}   
\caption{Simulations for the ArithMean Case (Saturation legend in Fig.~\ref{fig:ArithMeanCase_Saturation}(a) is representative for all the other figures - \ie Fig.~\ref{fig:ArithMeanCase_Saturation}(b) to (f))}
\label{fig:ArithMeanCase_Saturation}
\end{figure}
\end{landscape}
\clearpage



%%%%
%%%%  FIGURE
%%%%
\begin{landscape}
\begin{figure}[ht] 
\vbox{\vspace{-1cm}
\hspace{0.0cm} \hbox{\includegraphics[width=.56\textwidth]{./Pics/HarmMeanCase/HarmMeanCase_Saturation_t_dot15.png}
      \includegraphics[width=.56\textwidth]{./Pics/HarmMeanCase/HarmMeanCase_Saturation_t_dot30.png}
      \includegraphics[width=.56\textwidth]{./Pics/HarmMeanCase/HarmMeanCase_Saturation_t_dot50.png}}
\vspace{0.cm}
\hbox{\hspace{0.5cm} (a) Phase 1 Saturation at t=0.15s \hspace{3.75cm} (b) t=0.30s \hspace{5.0cm} (c) t=0.50s}
\vspace{0.5cm}
\hbox{
      \includegraphics[width=.56\textwidth]{./Pics/HarmMeanCase/HarmMeanCase_Saturation_t_1dot15.png}
      \includegraphics[width=.56\textwidth]{./Pics/HarmMeanCase/HarmMeanCase_Saturation_t_1dot75.png} 
      \includegraphics[width=.56\textwidth]{./Pics/HarmMeanCase/HarmMeanCase_Saturation_t_2dot95.png}}
\vspace{0.cm}
\hbox{ \hspace{2.5cm} (d) t=1.15s \hspace{5.5cm} (e) t=1.75s   \hspace{5.5cm} (f) t=2.95s}
\vspace{0.cm}
}   
\caption{Simulations for the HarmMean Case (Saturation legend in Fig.~\ref{fig:HarmMeanCase_Saturation}(a) is representative for all the other figures - \ie Fig.~\ref{fig:HarmMeanCase_Saturation}(b) to (f))}
\label{fig:HarmMeanCase_Saturation}
\end{figure}
\end{landscape}
\clearpage





%%%%
%%%%  FIGURE
%%%%
\begin{landscape}
\begin{figure}[ht] 
\vbox{\vspace{-1cm}
\hspace{0.0cm} \hbox{\includegraphics[width=.56\textwidth]{./Pics/PDFCase/PDFCase_Saturation_t_dot15.png}
      \includegraphics[width=.56\textwidth]{./Pics/PDFCase/PDFCase_Saturation_t_dot30.png}
      \includegraphics[width=.56\textwidth]{./Pics/PDFCase/PDFCase_Saturation_t_dot50.png}}
\vspace{0.cm}
\hbox{\hspace{0.5cm} (a) Phase 1 Saturation at t=0.15s \hspace{3.75cm} (b) t=0.30s \hspace{5.0cm} (c) t=0.50s}
\vspace{0.5cm}
\hbox{
      \includegraphics[width=.56\textwidth]{./Pics/PDFCase/PDFCase_Saturation_t_1dot15.png}
      \includegraphics[width=.56\textwidth]{./Pics/PDFCase/PDFCase_Saturation_t_1dot75.png} 
      \includegraphics[width=.56\textwidth]{./Pics/PDFCase/PDFCase_Saturation_t_2dot95.png}}
\vspace{0.cm}
\hbox{ \hspace{2.5cm} (d) t=1.15s \hspace{5.5cm} (e) t=1.75s   \hspace{5.5cm} (f) t=2.95s}
\vspace{0.cm}
}   
\caption{Simulations for the PDFCase (Saturation legend in Fig.~\ref{fig:PDFCase_Saturation}(a) is representative for all the other figures - \ie Fig.~\ref{fig:PDFCase_Saturation}(b) to (f))}
\label{fig:PDFCase_Saturation}
\end{figure}
\end{landscape}
\clearpage



%%%%
%%%%  FIGURE
%%%%
\begin{landscape}
\begin{figure}[ht] 
\vbox{\vspace{-1cm}
\hspace{0.0cm} \hbox{\includegraphics[width=.56\textwidth]{./Pics/SVDCase/SVDCase_Saturation_t_dot15.png}
      \includegraphics[width=.56\textwidth]{./Pics/SVDCase/SVDCase_Saturation_t_dot30.png}
      \includegraphics[width=.56\textwidth]{./Pics/SVDCase/SVDCase_Saturation_t_dot50.png}}
\vspace{0.cm}
\hbox{\hspace{0.5cm} (a) Phase 1 Saturation at t=0.15s \hspace{3.75cm} (b) t=0.30s \hspace{5.0cm} (c) t=0.50s}
\vspace{0.5cm}
\hbox{
      \includegraphics[width=.56\textwidth]{./Pics/SVDCase/SVDCase_Saturation_t_1dot15.png}
      \includegraphics[width=.56\textwidth]{./Pics/SVDCase/SVDCase_Saturation_t_1dot75.png} 
      \includegraphics[width=.56\textwidth]{./Pics/SVDCase/SVDCase_Saturation_t_2dot95.png}}
\vspace{0.cm}
\hbox{ \hspace{2.5cm} (d) t=1.15s \hspace{5.5cm} (e) t=1.75s   \hspace{5.5cm} (f) t=2.95s}
\vspace{0.cm}
}   
\caption{Simulation for the SVDCase (Saturation legend in Fig.~\ref{fig:SVDCase_Saturation}(a) is representative for all the other figures - \ie Fig.~\ref{fig:SVDCase_Saturation}(b) to (f))}
\label{fig:SVDCase_Saturation}
\end{figure}
\end{landscape}
\clearpage



%%%%
%%%%  FIGURE
%%%%
\begin{landscape}
\begin{figure}[ht] 
\vbox{\vspace{-1cm}
\hspace{0.0cm} \hbox{\hspace{4cm} \includegraphics[width=.56\textwidth]{./Pics/BaseCase/BaseCase_Saturation_t_1dot15withlegend.png}
      \hspace {1.5cm} \includegraphics[width=.56\textwidth]{./Pics/ArithMeanCase/ArithMeanCase_Saturation_t_1dot15.png}}
\vspace{0.cm}
\hbox{\hspace{7.0cm} (a) BaseCase \hspace{5.75cm} (b) ArithMeanCase}
\vspace{0.5cm}
\hbox{
      \includegraphics[width=.56\textwidth]{./Pics/HarmMeanCase/HarmMeanCase_Saturation_t_1dot15.png}
      \includegraphics[width=.56\textwidth]{./Pics/PDFCase/PDFCase_Saturation_t_1dot15.png} 
      \includegraphics[width=.56\textwidth]{./Pics/SVDCase/SVDCase_Saturation_t_1dot15.png}}
\vspace{0.cm}
\hbox{ \hspace{1.75cm} (c) HarmMeanCase \hspace{5.0cm} (d) PDFCase \hspace{5.0cm} (e) SVDCase}
\vspace{0.cm}
}   
\caption{Comparing phase 1 saturation distribution for all the models at $t=1.15$ s (Saturation legend in Fig.~\ref{fig:Saturationfield@t=1.15s}(a) is representative for all the other figures - \ie Fig.~\ref{fig:Saturationfield@t=1.15s}(b) to (e))}
\label{fig:Saturationfield@t=1.15s}
\end{figure}
\end{landscape}
\clearpage

%%%%
%%%%  FIGURE
%%%%
%\begin{landscape}

%\begin{figure}[ht] 
%\vbox{\vspace{-1cm}
%\centering
%\includegraphics[width=.56\textwidth]{./Pics/Introduction/Graphical-illustration-of-model-order-reduction.png}
%\vspace{0.cm}
%\vspace{0.5cm}
%}   
%\caption{Graphical illustration of model order reduction (initially used by \citet{Schilders2008}, with graphics credited to Harvard University, Microsoft Research.)}
%\vspace{1.5cm}
%\label{fig:IllustrationMOR}
%\end{figure}

\begin{figure}[ht] 
\vbox{\vspace{-1cm}
\hbox{\includegraphics[width=.56\textwidth]{./Pics/3D_BaseCase/3D_BaseCase_MeshOnly.png}
      \includegraphics[width=.56\textwidth]{./Pics/3D_HOSVDCase/3D_HOSVDCase_MeshOnly.png}}
\vspace{0.cm}
\hbox{\hspace{0.05cm} (a) 3-D High Resolution Grid (BaseCase) \hspace{0.05cm} (b) 3-D Low Resolution Grid (Upscaled Cases) \hspace{3.0cm}}
\vspace{0.5cm}
}   
\caption{Mesh grid used in the performed numerical 3-D simulations: (a) high-resolution and (b) upscaled cases with 2019 and 303 tetrahedral elements respectively}
\label{fig:HiRes_LowRes_3D_Mesh}
\end{figure}


%\end{landscape}
\clearpage

%%%%
%%%%  FIGURE
%%%%
\begin{landscape}
\begin{figure}[ht] 
\vbox{\vspace{-1cm}
\hspace{0.0cm} \hbox{\includegraphics[width=.8\textwidth]{./Pics/3D_BaseCase/3D_BaseCase_PermField_withMesh2.png}
      \includegraphics[width=.7\textwidth]{./Pics/3D_BaseCase/3D_BaseCase_PermField_withoutMesh.png}}
\vspace{0.cm}
\hbox{(a) 3-D BaseCase (overlapped with the mesh) \hspace{3.75cm} (b) 3-D BaseCase (without Mesh)}
\vspace{0.5cm}
\hbox{\includegraphics[width=.7\textwidth]{./Pics/3D_HOSVDCase/3D_HOSVDCase_PermField_withMesh.png}
      \includegraphics[width=.7\textwidth]{./Pics/3D_HOSVDCase/3D_HOSVDCase_PermField_withoutMesh.png}}
\vspace{0.cm}
\hbox{(c) 3-D HOSVD Case (overlapped with the mesh) \hspace{4.75cm} (d) 3-D HOSVD Case (without Mesh)}
\vspace{0.cm}
}   
\caption{Permeability field for the 3D base case as well as the HOSVD upscaled cases (Permeability legend in Fig.~\ref{fig:3D_PermFields}(a) is representative for all the other figures - \ie Fig.~\ref{fig:3D_PermFields}(b) to (d))}
\label{fig:3D_PermFields}
\end{figure}
\end{landscape}
\clearpage

%%%%
%%%%  FIGURE
%%%%
\begin{landscape}
\begin{figure}[ht] 
\vbox{\vspace{-1cm}
\hspace{0.0cm} \hbox{\includegraphics[width=.6\textwidth]{./Pics/3D_BaseCase/3D_BaseCase_Saturation_tdot18withMesh2.png}
      \includegraphics[width=.5\textwidth]{./Pics/3D_BaseCase/3D_BaseCase_Saturation_tdot28.png}
      \includegraphics[width=.5\textwidth]{./Pics/3D_BaseCase/3D_BaseCase_Saturation_tdot42.png}}
\vspace{0.cm}
\hbox{\hspace{0.5cm} (a) 3-D BaseCase at t=0.18s \hspace{1.0cm} (b) 3-D BaseCase at t=0.28s \hspace{3.0cm} (c) 3-D BaseCase at t=0.42s}
\vspace{0.5cm}
\hbox{
      \includegraphics[width=.5\textwidth]{./Pics/3D_HOSVDCase/3D_HOSVDCase_Saturation_tdot18withMesh.png}
      \includegraphics[width=.5\textwidth]{./Pics/3D_HOSVDCase/3D_HOSVDCase_Saturation_tdot28.png} 
      \includegraphics[width=.5\textwidth]{./Pics/3D_HOSVDCase/3D_HOSVDCase_Saturation_tdot42.png}}
\vspace{0.cm}
\hbox{ \hspace{1.5cm} (d) HOSVD Case at t=0.18s \hspace{1.0cm} (e) HOSVD Case at t=0.28s  \hspace{3.0cm} (f) HOSVD Case at t=0.42s}
\vspace{0.cm}
}   
\caption{Comparing phase 1 saturation distribution for the 3-D Cases at $t=0.18, 0.28$ and $0.42$ s (Saturation legend in Fig.~\ref{fig:Saturationfield4_3DCases}(a) is representative for all the other figures - \ie Fig.~\ref{fig:Saturationfield4_3DCases}(b) to (e))}
\label{fig:Saturationfield4_3DCases}
\end{figure}
\end{landscape}
\clearpage



%

%%%
%%%  FIGURE 
%%%
\begin{figure}[h]
\begin{center}
\includegraphics[width=1.\textwidth]{diagrams/bl-exact-meth-upwind.eps}
\end{center}
\caption{Buckley--Leverett test-cases: Saturation solutions for the continuous upwind method for different 1D P$_{1}$DG-P$_{2}$ mesh  resolutions and comparison against standard analytical solution.
\label{bl-exact-meth-upwind}}
\end{figure}

%%%
%%%
%%%  FIGURE 
%%%
\begin{figure}[h]
  %\begin{center}
\vbox{\hbox{\hspace{2.5cm}
    \includegraphics[width=0.62\textwidth]{diagrams/BL_1d_P0DGP1_convergence.eps}}
\vspace{-.0cm}\hbox{\hspace{2.5cm}
    \includegraphics[width=0.62\textwidth]{diagrams/BL_1d_P1DGP2_convergence.eps}}
\vspace{-.0cm}\hbox{\hspace{2.5cm}
    \includegraphics[width=0.62\textwidth]{diagrams/BL_1d_P2DGP3_convergence.eps}}}
   % \includegraphics[width=0.45\textwidth]{BL_2d_P1DGP2_convergence}
    \caption{Buckley--Leverett test-cases: Saturation profiles for a number of element-pairs and numerical resolutions in 1D -- P$_{0}$DG-P$_{1}$ (top), P$_{1}$DG-P$_{2}$ and P$_{2}$DG-P$_{3}$ (bottom).\label{fig:BL_profiles}}
  %\end{center}
\end{figure}

%%%
%%%  FIGURE 
%%%
\begin{figure}[h]
\vbox{\hbox{\hspace{1.cm}
    \includegraphics[width=0.8\textwidth]{diagrams/L1_convergence_rate.eps}}
\vspace{.0cm}\hbox{\hspace{1.cm}
    \includegraphics[width=0.8\textwidth]{diagrams/L2_convergence_rate.eps}}}
    \caption{Buckley--Leverett test-cases: L1 (top) and L2 (bottom) error convergence rates for a number of element pairs. \label{fig:BL_converg-rates}}
\end{figure}

%%%
%%%  FIGURE 
%%%
\begin{figure}[h]
\begin{center}
\includegraphics[width=1.\textwidth]{diagrams/bl-dg-2eles.eps}
\end{center}
\caption{Buckley--Leverett test-cases: Two element solution using the discontinuous formulation. Saturation field from both CV solution and FEM interpolation are shown.  \label{bl-dg-2eles}}
\end{figure}


%%%
%%%  FIGURE 
%%%
\begin{figure}[h]
\vbox{\hbox{\hspace{1.cm}
    \includegraphics[width=0.8\textwidth]{diagrams/L1_convergence_rate_DG.eps}}
\vspace{.0cm}\hbox{\hspace{1.cm}
    \includegraphics[width=0.8\textwidth]{diagrams/L2_convergence_rate_DG.eps}}}
    \caption{Buckley-Leverett test-cases: L1 (top) and L2 (bottom) error convergence rates for a number of fully discontinuous (between elements) element pairs. \label{fig:BL_converg-rates_DG}}
\end{figure}


%%%
%%%  FIGURE 
%%%
\begin{figure}[h]
\vbox{
\hbox{\hspace{.3cm}\includegraphics[width=.9\textwidth]{diagrams/bl-dg-cent-4-10-20.eps}}
\vspace{-0.cm}
\hbox{\hspace{.3cm}\includegraphics[width=.9\textwidth]{diagrams/bl-dg-4-10-20.eps}}}
\caption{Buckley--Leverett test-cases: Saturation field obtained from the discontinuous and continuous formulations with different mesh resolutions. Solutions without (top) and with (bottom) upwinding scheme. Notice that oscillations are suppressed with the upwinding scheme.\label{bl-dg-cent-4-10-20}}
\end{figure}


%%%
%%%  FIGURE 
%%%
\begin{figure}[h]
\vbox{
\hbox{\hspace{.3cm}\includegraphics[width=.9\textwidth]{diagrams/bl-dg-4-10-vers-cty.eps}}
\vspace{-0.cm}
\hbox{\hspace{.3cm}\includegraphics[width=.9\textwidth]{diagrams/bl-dg-p1-2-4-5-10-20-40.eps}}}
\caption{Buckley--Leverett test-cases: Saturation field obtained from (top) continuous and discontinuous (between elements) formulations (solution with 50 elements may be considered as a converged result). Solution obtained (bottom) from linear pressure $\left(\text{P}_{1}\right)$ formulation with different mesh resolution with comparison against P$_{2}$-pressure formulation (continuous). \label{bl-dg-4-10-vers-cty}}
\end{figure}

%%%
%%%  FIGURE 
%%%
\begin{figure}[h]
\vbox{
\hbox{\hspace{.2cm}
    \includegraphics[width=1.\textwidth]{diagrams/map_2d.png}}
\vspace{1.cm}
\hbox{\hspace{0.2cm}
    \includegraphics[width=1.\textwidth]{./diagrams/map_3d.png}}}
    \caption{Buckley-Leverett test-cases: phase 1 saturation surface maps for a 2- (770 triangles) and 3-D (1207 tetrahedra) simulations (\PN[1]{2} unstructured mesh grids) at time $t=0.5$. \label{fig:maps2d_3d}}
\end{figure}

%%%
%%%  FIGURE 
%%%
\begin{figure}[h]
\vbox{\hbox{\hspace{.3cm}
    \includegraphics[width=0.9\textwidth]{diagrams/BL_2d_P1DGP2_convergence.eps}}
\vspace{-.0cm}\hbox{\hspace{.3cm}
    \includegraphics[width=0.9\textwidth]{./diagrams/simulations_2d_3d.eps}}}
    \caption{Buckley-Leverett test-cases: 2- and 3-D phase 1 saturation profiles with \PN[1]{2} elements. Sensitivity analysis for (top) grid resolution using structured \PN[1]{2} mesh, and (bottom) mesh type.\label{fig:BL_2d_profiles}}
\end{figure}




\begin{comment}
%%%
%%%  FIGURE 
%%%
\begin{figure}[h]
\begin{center}
\includegraphics[width=1.\textwidth]{diagrams/bl-upwind-v-up-and-down.eps}
\end{center}
\caption{Buckley--Leverett test-cases: Comparison of the optimal upwind formulation when using upwinding (OU) and coupled upwind/downwind (OU-D). The finite element interpolation of the saturation field $\left(S_{1}\right)$ is shown at different mesh resolutions. Downwind seems to detract from the accuracy of the solution. \label{bl-upwind-v-up-and-down}}
\end{figure}

%%%
%%%  FIGURE 
%%%
\begin{figure}[h]
\vbox{
\begin{center}
\includegraphics[width=1.\textwidth]{diagrams/bl-exact-meth-cv-0-8-ele50.eps}
\end{center}
\vspace{0.cm}}
\caption{Buckley--Leverett test-cases: Comparison of control volume
  solutions using 80$\%$ upwinding and with optimal upwinding and
  using 50 continuous P$_{1}$DG-P$_{2}$
  elements. \label{bl-exact-meth-cv-0-8-ele50}}
\end{figure}



%%%
%%%  FIGURE 
%%%
\begin{figure}[H]
\vbox{
\begin{center}
\includegraphics[width=17.5cm,height=12.5cm]{diagrams/bl-dg-4-10-vers-cty}
\end{center}
\vspace{0.cm}}
\caption{Gas saturations shown comparing the accuracy of the
  discontinuous between elements and continuous formulation. The 50
  element continuous solution may be viewed as a converged result.  }
\label{bl-dg-4-10-vers-cty}
\end{figure}

%\begin{comment}

\end{comment}

%\end{comment}
 

\end{document}
%% End of tex file.


