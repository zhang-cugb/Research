%%%%%%%%%%%%%%%%%%%%%%%%%%%%%%%%%%%%%%%%%
% Plain Cover Letter
% LaTeX Template
% Version 1.0 (28/5/13)
%
% This template has been downloaded from:
% http://www.LaTeXTemplates.com
%
% Original author:
% Rensselaer Polytechnic Institute 
% http://www.rpi.edu/dept/arc/training/latex/resumes/
%
% License:
% CC BY-NC-SA 3.0 (http://creativecommons.org/licenses/by-nc-sa/3.0/)
%
%%%%%%%%%%%%%%%%%%%%%%%%%%%%%%%%%%%%%%%%%

%----------------------------------------------------------------------------------------
%	PACKAGES AND OTHER DOCUMENT CONFIGURATIONS
%----------------------------------------------------------------------------------------

\documentclass[11pt]{letter} % Default font size of the document, change to 10pt to fit more text
\usepackage{hyperref,url}
\hypersetup{colorlinks=true, urlcolor=blue, linkcolor=blue, citecolor=red}
%\usepackage{newcent} % Default font is the New Century Schoolbook PostScript font 
%\usepackage{helvet} % Uncomment this (while commenting the above line) to use the Helvetica font
%% The amssymb package provides various useful mathematical symbols

\usepackage{amssymb,amsmath,array}
%The amsthm package provides extended theorem environments

\usepackage{amsthm}
\usepackage{graphicx}
%\usepackage{subfigure}

\newcommand{\JGnote}[1]{\fbox{\parbox{\textwidth}{ \color{red} JG Note $\Rightarrow$ #1}}}
\newcommand{\BLnote}[1]{\fbox{\parbox{\textwidth}{ \color{green} BL Note $\Rightarrow$ #1}}}
\newcommand{\red}{\textcolor{red}}
\newcommand{\blue}{\textcolor{blue}}
\newcommand{\green}{\textcolor{green}}
\newcommand{\yellow}{\textcolor{yellow}}
\newcommand{\frc}{\displaystyle\frac}
\newcommand{\PN}[2][error]{P$_{#1}$DG-P$_{#2}$}
\newcommand{\PNDG}[2][error]{P$_{#1}$DG-P$_{#2}$DG}
\newcommand{\eg}{{\it e.g., }} 
\newcommand{\ie}{{\it i.e., }}
\newcommand{\st}{{\it s.t., }}



% Margins
\topmargin=-1in % Moves the top of the document 1 inch above the default
\textheight=8.5in % Total height of the text on the page before text goes on to the next page, this can be increased in a longer letter
\oddsidemargin=-10pt % Position of the left margin, can be negative or positive if you want more or less room
\textwidth=6.5in % Total width of the text, increase this if the left margin was decreased and vice-versa

%\let\raggedleft\raggedright % Pushes the date (at the top) to the left, comment this line to have the date on the right


%----------------------------------------------------------------------------------------
%	YOUR NAME & ADDRESS SECTION
%----------------------------------------------------------------------------------------

%\begin{center}
\address {Room 354, \\Fraser Noble Building \\ University of Aberdeen,\\ Aberdeen AB24 3UE. \\
  email: lashorebabatunde@yahoo.com} % Your address and phone number
%\end{center} 
%\vfill

\signature{B. Lashore} % Your name for the signature at the bottom
\longindentation=290pt
\begin{document}

%----------------------------------------------------------------------------------------
%	ADDRESSEE SECTION
%----------------------------------------------------------------------------------------

\begin{letter}{Professor Johann Sienz, \\CEng FIMechE, FRAeS, CMath, \\Editor-in-Chief of the International Journal: \\Applied Mathematical Modelling \\Deputy Head of College and, \\Director of Innovation and Engagement, \\
Swansea University, UK.}





%----------------------------------------------------------------------------------------
%	LETTER CONTENT SECTION
%----------------------------------------------------------------------------------------

\opening{Dear Professor Sienz,} 
 
The authors appreciate the reviewer's comments and have worked very hard to address all suggestions and comments.

The revised manuscript was just submitted to the journal's portal. Also, please find below the response to the reviewer's comments , highlighting the main changes in the manuscript.

\closing{Sincerely yours,}


\ps{
Editor: \\
The bibliography is not in the required format for Applied Mathematical Modelling. Please follow the author guide.
\url{https://www.elsevier.com/journals/applied-mathematical-modelling/0307-904x/guide-for-authors}}

Reviewer \#1: \\
The authors aim at reducing the computational cost of reservoir simulation by parameterizing the permeability field using SVD and they compare with the classical upscaling techniques. This very same topic has been investigated by several authors before and I do not see anything new in this presentation. It uses a very simplistic approach and many important questions are left in this paper. Moreover, the authors missed many of the important developments done in SVD-based and other forms of permeability  parameterization in recent years. The authors are referred to many publications listed here.\\
The paper also lacks comparison in computational effort, which is the main objective of such parameterization. For all of these reasons I am rejecting this paper for further publication.\\
$[1]$ Insuasty, E., Van den Hof, P. M. J., Weiland, S., \& Jansen, J. D. (2017, February 20). Low-Dimensional Tensor Representations for the Estimation of Petrophysical Reservoir Parameters. Society of Petroleum Engineers. doi:10.2118/182707-MS\\
$[2]$ Jafarpour, B., \& McLaughlin, D. B. (2009, March 1). Reservoir Characterization With the Discrete Cosine Transform. Society of Petroleum Engineers. doi:10.2118/106453-PA\\
$[3]$ Sardar Afra, Eduardo Gildin, Tensor based geology preserving reservoir parameterization with Higher Order Singular Value Decomposition (HOSVD), Computers \& Geosciences, Volume 94, 2016, Pages 110-120,\\
$[4]$  Tavakoli, R., \& Reynolds, A. C. (2010, June 1). History Matching With Parametrization Based on the SVD of a Dimensionless Sensitivity Matrix. Society of Petroleum Engineers. doi:10.2118/118952-PA\\

\encl{Cover\_letter\_for\_resubmission.pdf, Highlight\_revised.pdf, Manuscript\_sourceFiles(Makefile, Stochastic\_Upscaling\_UK\_AMM\_2018\_revised.tex, article\_/figure1.tex, Pics Folder, references.bib, elsarticle.cls, elsarticle-harv.bst)} % List your enclosed documents here, comment this out to get rid of the "encl:"

%----------------------------------------------------------------------------------------

\end{letter}

\end{document}
