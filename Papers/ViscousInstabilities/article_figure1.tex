
%%%%%%%%%%%%%%%%
%%% Figure 1 %%%
%%%%%%%%%%%%%%%%
%\begin{figure}[h]
%\begin{center}
%\includegraphics[width=.5\textwidth]{./Pics/P1DGP2.pdf}
%\caption{2D representation of the $P_{1}DGP_{2}$ element pairs used in this work. Shaded areas denote control volumes (in which saturation field is stored), blue points represent the pressure nodes and white points the velocity nodes.}
%\label{fig:fem_cv}
%\end{center}
%\end{figure}
%\clearpage

%%%%%%%%%%%%%%%%
%%% Figure 2 %%%
%%%%%%%%%%%%%%%%
%\begin{figure}[h] 
%\begin{center}
%\includegraphics[width=.5\textwidth]{./Pics/element_n.pdf}
%\caption{Porosity $\phi_{i}$, and permeability $K_{i}$ are associated to the elements. The velocity and pressure are associated to the element corners and the mid-points as described in fig.\ref{fig:fem_cv}; three CVs fit into a $2$D element. Saturation and saturation-dependent variables are stored CV-wise.}
%\label{fig:fem_elem}
%\end{center}
%\end{figure}
%\clearpage

%%%%
%%%%  FIGURE 
%%%%
\begin{figure}[h]
\centering
\vbox{\includegraphics[width=.5\textwidth]{./Pics/P1DGP2.pdf}}
\caption{2D representation of the $P_{1}DGP_{2}$ element pairs used in this work. Shaded areas denote control volumes (in which saturation field is stored), blue points represent the pressure nodes and white points the velocity nodes.}
\label{fig:fem_cv}
\end{figure}

%%%%
%%%%  FIGURE
%%%%
\begin{figure}[h]
\centering
\vbox{\includegraphics[width=.5\textwidth]{./Pics/element_n.pdf}}
\caption{This is a description of two different element types. Element A is the $P_{1}DGP_{2}$, while element B is the $P_{1}DGP_{1}$. Porosity $\phi_{i}$, and permeability $K_{i}$ are associated to the elements $P_{o}DG$. The velocity and pressure are associated to the element nodes and the mid-points as described in Fig.\ref{fig:fem_cv}; three CVs fit into a $2$D element. Saturation and saturation-dependent variables are stored CV-wise. }
\label{fig:fem_elem}
\end{figure}

%%%%
%%%%  FIGURE 1
%%%%
\begin{figure}[h]
\centering
\vbox{\includegraphics[width=0.75\textwidth]{./Pics/phase_vol_frac_uni_perm_1.pdf}}
\caption{Schematics of formation of flow instabilities during injection of a pure low viscosity fluid (red) into a domain saturated with a second fluid (dark blue). The ratio of viscosity between the two fluids is 5. In this case, the initially piston shape front collapses leading to the formation of several fingers.}
\label{fig:simple_case}
\end{figure}
\clearpage

%%%%
%%%%  FIGURE
%%%%
\begin{figure}[h]
\vbox{
\hbox{\hspace{2.5cm}
\includegraphics[width=.6\textwidth]{./Pics/2b2_P1DGP2.pdf} 
}
\vspace{0.0cm}
\hbox{\hspace{6.0cm} (a)  
}
\vspace{0.5cm}
\hbox{\hspace{2.5cm}
\includegraphics[width=.6\textwidth]{./Pics/2b2_P1DGP2_plot.pdf}
}
\vspace{1.0cm}
\hbox{\hspace{6.0cm} (b)  
}}     
\caption{2D representation of the element pairs presented in this work. Shaded areas denote control volumes (in which saturation is stored), black points represent the pressure nodes and white points the velocity nodes. Note that in (b) velocity and pressure nodes overlap in the triangles' vertices.}
\label{fem_cv_represent_a}
\end{figure}
\clearpage

%%%%
%%%%  FIGURE
%%%%
\begin{figure}[ht] 
\vbox{
\hbox{\hspace{0.5cm}
\includegraphics[width=.85\textwidth]{./Pics1/2b2_wi_fine/2b2_whole_in_fine_perm_1.pdf} 
}
\vspace{0.0cm}
\hbox{\hspace{5.0cm} (a) map of permeabilties K  
}
\vspace{0.25cm}
\hbox{\hspace{1.5cm}
\includegraphics[width=.75\textwidth]{./Pics1/2b2_wi_fine/2b2_whole_in_fine_250_2.pdf}
}
\vspace{0.0cm}
\hbox{\hspace{5.0cm} (b) flow at t=250  
}
\vspace{0.25cm}
\hbox{\hspace{1.5cm}
\includegraphics[width=.75\textwidth]{./Pics1/2b2_wi_fine/2b2_whole_in_fine_3000_1.pdf}
}
\vspace{0.0cm}
\hbox{\hspace{5.0cm} (c) flow at t=3000  
}
}     
\caption{Using unstructured but fixed mesh the domain is devide into $4$ different regions of different permeabilities as these are represented in the figures above}
\label{fem_cv_represent_a}
\end{figure}
\clearpage

%%%%
%%%%  FIGURE
%%%%
\begin{figure}[ht] 
\vbox{
\hbox{\hspace{1.5cm}
\includegraphics[width=.85\textwidth]{./Pics1/mr1_fixed/mr1_fixed_100_2.pdf} 
}
\vspace{0.0cm}
\hbox{\hspace{4.0cm} (a) MR1 at t = 100 (start)   
}
\vspace{0.25cm}
\hbox{\hspace{1.5cm}
\includegraphics[width=.85\textwidth]{./Pics1/mr10_fixed/mr10_fixed_100_1.pdf}
}
\vspace{0.0cm}
\hbox{\hspace{4.0cm} (b) MR10 at t = 100 (start)  
}
}     
\caption{Comparing the the MR1 and MR10 cases using fixed mesh in the beggining of the simulation.}
\label{fig:6a}
\end{figure}
\clearpage

%%%%
%%%%  FIGURE
%%%%
\begin{figure}[ht] 
\vbox{
\hbox{\hspace{1.5cm}
\includegraphics[width=.85\textwidth]{./Pics1/mr1_fixed/mr1_fixed_middle_1.pdf} 
}
\vspace{0.0cm}
\hbox{\hspace{4.0cm} (a) MR1 at t = 1500 (middle)   
}
\vspace{0.25cm}
\hbox{\hspace{1.5cm}
\includegraphics[width=.85\textwidth]{./Pics1/mr10_fixed/mr10_fixed_middle_1.pdf}
}
\vspace{0.0cm}
\hbox{\hspace{4.0cm} (b) MR10 at t = 1500 (middle)  
}
}     
\caption{Comparing the the MR1 and MR10 cases using fixed mesh in the middle of the simulation.}
\label{fig:6b}
\end{figure}
\clearpage

%%%%
%%%%  FIGURE
%%%%
\begin{figure}[ht] 
\vbox{
\hbox{\hspace{1.5cm}
\includegraphics[width=.85\textwidth]{./Pics1/mr1_fixed/mr1_fixed_middle_1.pdf} 
}
\vspace{0.0cm}
\hbox{\hspace{4.0cm} (a) MR1 at t = 3000 (end)   
}
\vspace{0.25cm}
\hbox{\hspace{1.5cm}
\includegraphics[width=.85\textwidth]{./Pics1/mr10_fixed/mr10_fixed_middle_1.pdf}
}
\vspace{0.0cm}
\hbox{\hspace{4.0cm} (b) MR10 at t = 3000 (end)  
}
}     
\caption{Comparing the the MR1 and MR10 cases using fixed mesh in the end of the simulation.}
\label{fig:6c}
\end{figure}
\clearpage

%%%%
%%%%  FIGURE
%%%%
\begin{figure}[ht] 
\vbox{
\hbox{\hspace{2.5cm}
\includegraphics[width=.6\textwidth]{./Pics1/Saffman_homogeneous_MR3/saffman_homo_fixed_1.pdf} 
}
\vspace{0.0cm}
\hbox{\hspace{4.0cm} (a) Hell-Shaw cell domain   
}
\vspace{0.25cm}
\hbox{\hspace{2.5cm}
\includegraphics[width=.6\textwidth]{./Pics1/Saffman_homogeneous_MR3/saffman_homo_fixed_250.pdf}
}
\vspace{0.0cm}
\hbox{\hspace{5.0cm} (b) flow at t=250  
}
}     
\caption{This is a description of a Hele-Shaw experimental domain, showing the source and sink term as this is implied in this test-case followed by a screenshot of the flow at time, t=250(b). More screenshot follow that describe the progression of the flow in figure \ref{fig:1b_homoheleshaw_3} and figure \ref{fig:1c_homoheleshaw_3}.}
\label{fig:1a_homoheleshaw_3}
\end{figure}
\clearpage

%%%%
%%%%  FIGURE
%%%%
\begin{figure}[ht] 
\vbox{
\hbox{\hspace{2.5cm}
\includegraphics[width=.6\textwidth]{./Pics1/Saffman_homogeneous_MR3/saffman_homo_fixed_1000.pdf} 
}
\vspace{0.0cm}
\hbox{\hspace{5.0cm} (a) flow at t=1000   
}
\vspace{0.25cm}
\hbox{\hspace{2.5cm}
\includegraphics[width=.6\textwidth]{./Pics1/Saffman_homogeneous_MR3/saffman_homo_fixed_2500.pdf}
}
\vspace{0.0cm}
\hbox{\hspace{5.0cm} (b) flow at t=2500  
}
}     
\caption{Following figure \ref{fig:1a_homoheleshaw_3} as the flow progress, t=1000(a), a tip start to appears, at t=2500(b).}
\label{fig:1b_homoheleshaw_3}
\end{figure}
\clearpage

%%%%
%%%%  FIGURE
%%%%
\begin{figure}[ht] 
\vbox{
\hbox{\hspace{2.5cm}
\includegraphics[width=.6\textwidth]{./Pics1/Saffman_homogeneous_MR3/saffman_homo_fixed_3500.pdf} 
}
\vspace{0.0cm}
\hbox{\hspace{5.0cm} (a) flow at t=3500   
}
\vspace{0.25cm}
\hbox{\hspace{2.5cm}
\includegraphics[width=.6\textwidth]{./Pics1/Saffman_homogeneous_MR3/saffman_homo_fixed_end_1.pdf}
}
\vspace{0.0cm}
\hbox{\hspace{5.0cm} (b) flow at t=end  
}
}     
\caption{This tip will become sharper at t=3500(a) as it approaches the area where sink term exist and finally at t=end(b) it will be absorbed.}
\label{fig:1c_homoheleshaw_3}
\end{figure}
\clearpage

%%%%
%%%%  FIGURE
%%%%
\begin{figure}[ht] 
\vbox{
\hbox{\hspace{2.5cm}
\includegraphics[width=.6\textwidth]{./Pics1/Saffman_homogeneous/saffman_homo_fixed_1.pdf} 
}
\vspace{0.0cm}
\hbox{\hspace{4.5cm} (a) Hell-Shaw cell domain   
}
\vspace{0.25cm}
\hbox{\hspace{2.5cm}
\includegraphics[width=.6\textwidth]{./Pics1/Saffman_homogeneous/saffman_homo_fixed_250_1.pdf}
}
\vspace{0.0cm}
\hbox{\hspace{5.0cm} (b) flow at t=250  
}
}     
\caption{This is a description of a Hele-Shaw experimental domain, when MR=$10$ showing the source and sink term as this is implied in this test-case followed by a screenshot of the flow at time, t=250(b). Screenshot will follow that describe the progression of the flow in figure \ref{fig:2b_homoheleshaw_10} and figure \ref{fig:2c_homoheleshaw_10}.}
\label{fig:2a_homoheleshaw_10}
\end{figure}
\clearpage

%%%%
%%%%  FIGURE
%%%%
\begin{figure}[ht] 
\vbox{
\hbox{\hspace{2.5cm}
\includegraphics[width=.6\textwidth]{./Pics1/Saffman_homogeneous/saffman_homo_fixed_1000.pdf} 
}
\vspace{0.0cm}
\hbox{\hspace{5.0cm} (a) flow at t=1000   
}
\vspace{0.25cm}
\hbox{\hspace{2.5cm}
\includegraphics[width=.6\textwidth]{./Pics1/Saffman_homogeneous/saffman_homo_fixed_4000.pdf}
}
\vspace{0.0cm}
\hbox{\hspace{5.0cm} (b) flow at t=4000  
}
}     
\caption{As the flow progress at t=1000(a) the front still remains uniform. The first break down of the front is happening at t=4000(b) where $3$ tips start to come out of the main flow.}
\label{fig:2b_homoheleshaw_10}
\end{figure}
\clearpage

%%%%
%%%%  FIGURE
%%%%
\begin{figure}[ht]
\vbox{
\hbox{\hspace{2.5cm}
\includegraphics[width=.6\textwidth]{./Pics1/Saffman_homogeneous/saffman_homo_fixed_6000.pdf} 
}
\vspace{0.0cm}
\hbox{\hspace{5.0cm} (a) flow at t=6000   
}
\vspace{0.25cm}
\hbox{\hspace{2.5cm}
\includegraphics[width=.6\textwidth]{./Pics1/Saffman_homogeneous/saffman_homo_fixed_end_1.pdf}
}
\vspace{0.0cm}
\hbox{\hspace{5.0cm} (b) flow at t=end   
}
}     
\caption{Finger are now more obvious as the flow progress and the tip-splitting mechanism becomes stronger at t=6000(a) as it approaches the area where sink term exist and finally at t=end(b) it will be absorbed.}
\label{fig:2c_homoheleshaw_10}
\end{figure}
\clearpage

%%%%
%%%%  FIGURE
%%%%
\begin{figure}[ht] 
\vbox{
\hbox{\hspace{2.5cm}
\includegraphics[width=.6\textwidth]{./Pics1/Saffman_heterogeneous/saffman_heter_fixed_1.pdf} 
}
\vspace{0.0cm}
\hbox{\hspace{4.5cm} (a) Hell-Shaw cell domain   
}
\vspace{0.25cm}
\hbox{\hspace{2.5cm}
\includegraphics[width=.6\textwidth]{./Pics1/Saffman_heterogeneous/saffman_heter_fixed_500.pdf}
}
\vspace{0.0cm}
\hbox{\hspace{5.0cm} (b) flow at t=250     
}
}     
\caption{This is a description of a heterogeneous Hele-Shaw experimental domain, when MR=$10$ showing the source and sink term as this is implied in this test-case followed by a screenshot of the flow at time, t=500(b). Screenshot will follow that describe the progression of the flow in figure \ref{fig:3b_heteheleshaw_10} and figure \ref{fig:3c_heteheleshaw_10}.}
\label{fig:3a_heteheleshaw_10}
\end{figure}
\clearpage

