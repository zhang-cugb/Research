% First of two papers based on the SPE Comparative Solution Project

%\documentclass{elsarticle}
\documentclass{article}
% package to allow adding affiliation into the title
\usepackage[affil-it]{authblk}
% todonotes allow adding notes into the text as comments
\usepackage{todonotes,amsmath,color}
\usepackage[normalem]{ulem}
\newcommand{\JG}[1]{\todo[color=blue!30, size=\small]{JG: #1}}
\newcommand{\red}{\textcolor{red}}


\begin{document}

\title{Reservoir Simulation using a Deterministic Heterogeneity generated from a discrete Permeability Data}


\author{B. Lashore\thanks{Electronic address: \texttt{r01bol14@abdn.ac.uk} (Corresponding author)} , J.L.M.A. Gomes}
\affil{School of Engineering, University of Aberdeen, UK}

\author{P. Salinas, C.C. Pain}\affil{Department of Earth Science and Engineering, Imperial College London, UK}
%\author[UoA]{B. Lashore}\corref{cor1}\ead{r01bol14@abdn.ac.uk}
%\author[IC]{P. Salinas} \author[IC]{C.C. Pain} \author[UoA]{J.L.M.A. Gomes}

%\cortext[cor1]{Corresponding author.}
%\address[UoA]{School of Engineering, University of Aberdeen, UK}
%\address[IC]{Department of Earth Science and Engineering, Imperial College London, UK}

\date \today
\maketitle
%\section*{Abstract}
\begin{abstract}

This paper presents an arguement for using a deterministic heterogeneous data for reservoir simulation, instead of the homogeneous data in use currently, within the Oil and Gas industry. It then goes on to describe a method of creating a deterministic heterogeneity from a set of discrete data for permeability. Finally, the paper discusses the outcome of a benchmark performed against other simulation results from the 10$^{\text{th}}$ SPE Comparative Solution Project (SPE 10 project). Although, only the first case from the SPE 10 project, which involves upscaling and pseudoization methods for an easily computed 2D gas-injection was simulated and benchmarked, for this paper.

The overlapping control volume finite element method (OCVFEM) formulation is capable of using a deterministic heterogeneous data for reservoir simulation. It employes triangular and tetrahedral finite element pairs in representing the model properties. Permeability is the property used to carry out the analysis discussed in this report. And, a P0DG meshing scheme is used for it in the simulation carried out. This means the nodal data point for permeability are defined as discontinous data points at the centre of each tethrahedral element. The permeability data for the deterministic heterogeneity is spatially assigned within the simulation domain through a two stage process. First, a suitable mesh size is selected which would allow the available discrete data, to be optimized for the second stage of the process. And the second stage is the determination of each nodal data value for the P0DG meshing scheme using 3D interpolation technique.

\red{Yet to obtain any information on the simulation}

\red{Try to highlight the preliminary results and what you expect to obtain 'till June/July.}

\red{Results, Conclusions. As we do not have them yet, you could briefly describe the preliminary results and the aim of validating the methods against the SPE10}.


This novel method of reservoir simulation does not require upscaling but instead makes use of the heterogeneous data available within the simulation domain. This has two advantages. First, it removes the need for a computation for upscaling. Second, flow simulation within the domain is similar to that of a real-life heterogeneous reservoir.

In a future paper, the second problem of the 10th SPE Comparative Solution Project would be simulated with the OCVFEM, and higher order meshing schemes would be used. The paper would also make use of stochastic methods for data representation. 
\end{abstract}

\red{\begin{center}\large{SPE Instructions}\end{center}
\begin{enumerate}
  \item Objectives/Scope: Please list the objectives and/or scope of the proposed paper. (25-75 words)
  \item Methods, Procedures, Process: Briefly explain your overall approach, including your methods, procedures and process. (75-100 words)
  \item Results, Observations, Conclusions: Please describe the results, observations and conclusions of the proposed paper. (100-200 words)
  \item Novel/Additive Information: Please explain how this paper will present novel (new) or additive information to the existing body of literature that can be of benefit to and/or add to the state of knowledge in the petroleum industry. (25-75 words) 
\end{enumerate}}

\end{document}
