% First of two papers based on the SPE Comparative Solution Project

%\documentclass{elsarticle}
\documentclass{article}
% package to allow adding affiliation into the title
\usepackage[affil-it]{authblk}
% todonotes allow adding notes into the text as comments
\usepackage{todonotes,amsmath,color}
\usepackage[normalem]{ulem}
\newcommand{\JG}[1]{\todo[color=blue!30, size=\small]{JG: #1}}
\newcommand{\red}{\textcolor{red}}

\begin{document}

\title{Reservoir Simulation using a Deterministic Heterogeneity generated from a discrete Permeability Data}


\author{B. Lashore\thanks{Electronic address: \texttt{r01bol14@abdn.ac.uk} (Corresponding author)} , J.L.M.A. Gomes}
\affil{School of Engineering, University of Aberdeen, UK}

\author{P. Salinas, C.C. Pain}\affil{Department of Earth Science and Engineering, Imperial College London, UK}
%\author[UoA]{B. Lashore}\corref{cor1}\ead{r01bol14@abdn.ac.uk}
%\author[IC]{P. Salinas} \author[IC]{C.C. Pain} \author[UoA]{J.L.M.A. Gomes}

%\cortext[cor1]{Corresponding author.}
%\address[UoA]{School of Engineering, University of Aberdeen, UK}
%\address[IC]{Department of Earth Science and Engineering, Imperial College London, UK}

\date \today
\maketitle

\begin{abstract}

This paper discusses the benefits of using a deterministic heterogeneous data for reservoir simulation, instead of a homogeneous data\JG{I don't understand this sentence. Homogenous data $\times$ heterogeneous data?}. Then it describes a method of creating a deterministic heterogeneity\JG{what do you mean w deterministic heterogeneity?}. Finally, it discusses the outcome of a benchmark performed against other simulation results from the 10$^{\text{th}}$ SPE Comparative Solution Project (SPE10 project)\JG{Suggestion: Finally, the XXX (what sort of solution) solution is benchmarked against the 10$^{\text{th}}$ SPE Comparative Solution Project (SPE10 project).}. Albeit, only the first case from the SPE10 Project, which involves upscaling and pseudoization methods for an easily computed 2D gas-injection\JG{I guess you missed the second part of the sentence}.

The overlapping control volume finite element method (OCVFEM) formulation is capable of using deterministic heterogeneity \red{?} for reservoir simulation. It employes \red{a dual consistent pressure-velocity representation in CV and FEM spaces embedded in a novel family of} \sout{triangular and} tetrahedral finite element-pairs \sout{in representing the model properties}. A P0DG \red{(you haven't defined this symbol ...)} meshing scheme is used \red{to represent the} \sout{for} absolute permeability tensor field \sout{, to perform this analysis}. This means the nodal data points are \red{discontinuously} defined \sout{as discontinous,} at the centre of each tetrahedral element. The deterministic heterogeneity is spatially assigned within the domain in two stages. First, a suitable mesh size is selected to optimize the use of the discrete data \red{you are being too vague ...}. Then, each nodal data value is determined using a 3D interpolation technique. 

The tetrahedral meshing produced more elements \sout{(cells)} than the given fine \red{hexahedral grid of SPE10 test} (100 x 1 x 20) in an effort to optimize the discrete data available. The interpolation scheme used allowed the permeability profile to change gradually between discrete data points. Three different mesh sizes were used to explain the relevance of optimizing the mesh size\red{(what do you mean?)}. And, it was shown that a mesh size that is relatively small compared to the distance between the nearest discrete data points produced unnecessarily large computational load for the simulator \red{(?)}. For the preliminary simulation result, as expected, the flow profile shows a high velocity in the high permeability region and a low velocity in the low permeability region. Therefore, the simulation gives a better understanding of the displacement front and helps to give a realistic view of the movement of the fluids in the reservoir.

This novel method of reservoir simulation does not require upscaling but instead makes use of the heterogeneous data available within the simulation domain. This has two advantages. First, it removes the need for a computation for upscaling. Second, flow simulation within the domain is similar to that of a real-life heterogeneous reservoir.\JG{(a) Try to first explain what you mean with deterministic heterogeneity! (b) You need to be more concise (i.e., less vague).}

\end{abstract}

\end{document}
